\documentclass[11pt,a4paper]{article}
\usepackage{isabelle,isabellesym}

% further packages required for unusual symbols (see also
% isabellesym.sty), use only when needed

%\usepackage{amssymb}
  %for \<leadsto>, \<box>, \<diamond>, \<sqsupset>, \<mho>, \<Join>,
  %\<lhd>, \<lesssim>, \<greatersim>, \<lessapprox>, \<greaterapprox>,
  %\<triangleq>, \<yen>, \<lozenge>

%\usepackage{eurosym}
  %for \<euro>

%\usepackage[only,bigsqcap]{stmaryrd}
  %for \<Sqinter>

%\usepackage{eufrak}
  %for \<AA> ... \<ZZ>, \<aa> ... \<zz> (also included in amssymb)

%\usepackage{textcomp}
  %for \<onequarter>, \<onehalf>, \<threequarters>, \<degree>, \<cent>,
  %\<currency>

% this should be the last package used
\usepackage{pdfsetup}

% urls in roman style, theory text in math-similar italics
\urlstyle{rm}
\isabellestyle{it}

% for uniform font size
%\renewcommand{\isastyle}{\isastyleminor}


\begin{document}

\title{Applying the Isabelle Insider Framework to Airplane Security}
\author{Florian Kamm\"uller and Manfred Kerber}

\maketitle

\begin{abstract}
Avionics is one of the fields in which verification methods have been pioneered 
and brought a new level of reliability to systems used in safety critical 
environments. Tragedies, like the 2015 insider attack on a German airplane, 
in which all 150 people on board died, show that safety and security crucially 
depend not only on the well functioning of systems but also on the way how 
humans interact with the systems. Policies are a way to describe how humans 
should behave in their interactions with technical systems, formal reasoning 
about such policies requires integrating the human factor into the 
verification process.

We model insider attacks on airplanes 
using logical modelling and analysis of infrastructure models 
and policies with actors to scrutinize security policies in the presence of 
insiders \cite{kk:16}. 
The Isabelle Insider framework has been first 
presented in \cite{kp:16}. 
Triggered by case studies, like the present one of airplane security, it 
has been greatly extended now formalizing Kripke structures and the temporal 
logic CTL to enable reasoning on dynamic system states. 
Furthermore, we illustrate that Isabelle modelling and invariant 
reasoning reveal subtle security assumptions: the formal development uses
locales to model the assumptions on insider and their access credentials.
Technically interesting is how the locale is interpreted in the presence
of an abstract type declaration for actor in the Insider framework redefining 
this type declaration at a later stage like a ``post-hoc type definition'' 
as proposed in \cite{mw:09}.
The case study and the application of the methododology are described in more 
detail in the preprint \cite{kk:20}.
\end{abstract}

\tableofcontents

% sane default for proof documents
\parindent 0pt\parskip 0.5ex

% generated text of all theories
\input{session}

% optional bibliography
\bibliographystyle{abbrv}
\bibliography{root}

\end{document}

%%% Local Variables:
%%% mode: latex
%%% TeX-master: t
%%% End:
